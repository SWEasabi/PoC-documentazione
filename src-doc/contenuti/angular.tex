\chapter{Angular}\label{angular}

\section{Introduzione}

La pagina Angular mostra due differenti bottoni, uno di Fetch dello stato della API e un altro che modifica lo stato dell'entità\footnote{nel dominio del POC, la prima entitá ID=1}, in questo bottone di modifica, il comportamento é quello di fetchare lo stato della lampada,per poi modificarlo nel suo complementare.
Vengono Dichiarati i componenti Angular "app" e "lamp-button", che inglobano rispettivamente i fogli di stile e Markup, oltre che i rispettivi file typescript "component.ts" e ".module.ts"\footnote{per il componente app}. 
Vengono dichiarati i moduli "BrowserModule", essenziale per lanciare una app browser, e "HttpClientModule", essenziale per fare richieste HTTP, è così possibile implementare una GET, POST, DELETE oppure PUT.
Nel Dominio del PoC si offrono tramite il metodo toggleLamp(), l'implementazione di una PUT per cambiare lo stato della lampada, innestato in una GET, capace di richiedere lo stato corrente della lampada stessa.

\subsection{getData}

Si offre il metodo getData, il quale da una chiamata GET all'url dello stato di tutte le lampade, quindi il loro id, il loro stato attuale\footnote{acceso = on, spento = off} e la loro locazione.

\subsection{toggleLamp}

Si offre il metodo toggleLamp(), tale metodo fa una chiamata "this.http.get<any>" su lamps/1, si aspetta una risposta di tipo JSON e porta tale risposta alla richiesta innestata "this.http.put<any>", la quale verifica lo stato del sotto campo dell'oggetto Response, Response.status, e lo cambia con il suo complementare\footnote{status: Response.status == 'on' ? 'off' : 'on'}.

Ritorna, alla fine della PUT, un Console.log con l'oggetto Response, che in questo momento temporale e' pari al risultato della operazione\footnote{e' stato fatto l'update, se modifica portata a buon fine}.


\begin{figure}[H]
    \centering
    \includegraphics[width=\linewidth]{getlamps.jpg}
    \caption{Button fa richiesta su /Lamps}
\end{figure}


\begin{figure}[H]
    \centering
    \includegraphics[width=\linewidth]{modifylamps.jpg}
    \caption{Button fa cambio di stato su /Lamps/1}
\end{figure}

\begin{figure}[H]
    \centering
    \includegraphics[width=0.3\linewidth]{modifylamps2.jpg}
    \caption{Button fa cambio di stato su /Lamps/1}
\end{figure}